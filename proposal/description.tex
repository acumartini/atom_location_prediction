\section{Project Description}
To tackle this problem we will run simulations of the CCD output for a given atom location by solving the linear wave equation using a Partial Differential Equation (PDE) Solver and/or a raytracing simulation.  The output images will be labeled by atom location and used as training set for machine learning algorithms.  The algorithms can then be used to generate a probabilistic distribution for the atom's location for a new CCD image.

%Diffyq is a distributed partial differential equation (PDE) solver for 3-dimensional models.  This systems takes advantage of the memory and processing power of multiple nodes to efficiently compute the PDE.  Given a 3D dataset, a user defined set of update equations, a time step granularity, and a number of steps; our system automatically marshalls the data and computational code to multiple nodes to initialize processing.  The PDE is then computed for the given number of steps and the final result of stored on the main node.

%A primary goal of this project to create a generalized application that can used to solve the PDE for any initial conditions.  To accomplish this, we allow the user to provide a set of user defined functions as input for computation.  Our system uses these functions to iteratively update the PDE matrix during parallel processing.  The details of the format for these user defined functions are described in Section \ref{architecture}.

%Communication between nodes allows the system to calculate the border values of between the divisions of the PDE matrix.  This is an example of the stenciling pattern for parallel processing.  On each node, our program takes advantage of opportunities for shared-memory parallelism through map-reduce and divide conquer techniques for computation update components.

%Since data output rates are often a limiting factor for the runtime of, we also will apply data reduction techniques to reduce the overall size of the data that will be written to disk. These include subsampling in both time and space, and compression methods (e.g. discrete cosine transform),
