\section{Parallel Plan}

The each stage of the project data flow includes several places to introduce and explore parallelism.

\begin{description}
\item[Simulation] PETSc...
\item[Machine Learning] There is at least one parallel machine learning library written in c++ that we can use as a starting point.  To get specialized functionality or algorithms that perform well for image recognition, we may need to extend existing libraries or code our own algorithms.  It is unclear whether a distributed program will be necessary to train the models given the number of features and size of the dataset.
\item[Visualization] Visit, paraview...

%\item[Data Reduction] Data reduction...
%\begin{enumerate}
%  \item Spacial subsampling
%  \item Temporal subsampling
%  \item Compression
%\end{enumerate}
\end{description}

%We plan to look at system scaling by varying the number of initial objects in the simulation and the number of cannons available. We intend to look at wall clock time to complete a fixed number of simulation steps. We also intend to look at latency between the FOSA transmission time and BDS receive time to see if pipelining provides a significant benefit.

%If we have extra time, we would like to explore using MPI to spread the simulator and AKWDS computations across nodes to increase the size of simulation that can be done in realtime.

