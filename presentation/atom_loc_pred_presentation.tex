%%%%%%%%%%%%%%%%%%%%%%%%%%%%%%%%%%%%%%%%%
% Beamer Presentation
% LaTeX Template
% Version 1.0 (10/11/12)
%
% This template has been downloaded from:
% http://www.LaTeXTemplates.com
%
% License:
% CC BY-NC-SA 3.0 (http://creativecommons.org/licenses/by-nc-sa/3.0/)
%
%%%%%%%%%%%%%%%%%%%%%%%%%%%%%%%%%%%%%%%%%

%----------------------------------------------------------------------------------------
%	PACKAGES AND THEMES
%----------------------------------------------------------------------------------------

\documentclass{beamer}
\setbeamertemplate{navigation symbols}{} % comment this out to ADD navigation tools

\usepackage{sansmathaccent}
\pdfmapfile{+sansmathaccent.map}

\mode<presentation> {

% The Beamer class comes with a number of default slide themes
% which change the colors and layouts of slides. Below this is a list
% of all the themes, uncomment each in turn to see what they look like.

%\usetheme{default}
%\usetheme{AnnArbor}
%\usetheme{Antibes}
%\usetheme{Bergen}
%\usetheme{Berkeley}
%\usetheme{Berlin}
%\usetheme{Boadilla}
%\usetheme{CambridgeUS}
%\usetheme{Copenhagen}
%\usetheme{Darmstadt}
%\usetheme{Dresden}
%\usetheme{Frankfurt}
%\usetheme{Goettingen}
%\usetheme{Hannover}
%\usetheme{Ilmenau}
%\usetheme{JuanLesPins}
%\usetheme{Luebeck}
%\usetheme{Madrid}
%\usetheme{Malmoe}
%\usetheme{Marburg}
%\usetheme{Montpellier}
%\usetheme{PaloAlto}
%\usetheme{Pittsburgh}
%\usetheme{Rochester}
%\usetheme{Singapore}
\usetheme{Szeged}
%\usetheme{Warsaw}

% As well as themes, the Beamer class has a number of color themes
% for any slide theme. Uncomment each of these in turn to see how it
% changes the colors of your current slide theme.

%\usecolortheme{albatross}
\usecolortheme{beaver}
%\usecolortheme{beetle}
%\usecolortheme{crane}
%\usecolortheme{dolphin}
%\usecolortheme{dove}
%\usecolortheme{fly}
%\usecolortheme{lily}
%\usecolortheme{orchid}
%\usecolortheme{rose}
%\usecolortheme{seagull}
%\usecolortheme{seahorse}
%\usecolortheme{whale}
%\usecolortheme{wolverine}

%\setbeamertemplate{footline} % To remove the footer line in all slides uncomment this line
%\setbeamertemplate{footline}[page number] % To replace the footer line in all slides with a simple slide count uncomment this line

%\setbeamertemplate{navigation symbols}{} % To remove the navigation symbols from the bottom of all slides uncomment this line
}

\usepackage{graphicx} % Allows including images
\usepackage{booktabs} % Allows the use of \toprule, \midrule and \bottomrule in tables
\usepackage{amssymb}
\usepackage{amsmath}
\usepackage{array}
\usepackage{longtable}
\usepackage{everysel}
\usepackage{xtab}
\usepackage{ragged2e}
\usepackage[labelfont=bf]{caption}
\usepackage{url}
\usepackage{datetime}
\usepackage{setspace}
\usepackage{multicol}
\usepackage{vwcol}
\usepackage{caption}
\usepackage{subcaption}

% texpos setup for placing text or images anywhere on the page
\usepackage[absolute,overlay]{textpos}
\setlength{\TPHorizModule}{30mm}
\setlength{\TPVertModule}{\TPHorizModule}
\textblockorigin{10mm}{10mm} % start everything near the top-left corner

% line spacing options
%\singlespacing
%\onehalfspacing
%\doublespacing
\setstretch{1.3} % for custom spacing


%---------------------------------------------------------------------------
%	TITLE PAGE
%---------------------------------------------------------------------------

\title[Predicting Atom Location Using Machine Learning Algorithms]{Predicting Atom Location Using \\ Machine Learning Algorithms} % The short title appears at the bottom of every slide, the full title is only on the title page

\author{Adam Martini, Ran Tian, Wes Erickson} % your name
\institute[UO] % Your institution as it will appear on the bottom of every slide, may be shorthand to save space
{
University of Oregon \\ % Your institution for the title page
\medskip
\textit{martini@cs.uoregon.edu\\} % Your email address
\textit{jmty0083@cs.uoregon.edu\\}
\textit{wwe@uoregon.edu}
}

\newdate{date}{13}{6}{2014}
\date{\displaydate{date}} % Date, can be changed to a custom date

\begin{document}

\begin{frame}
\titlepage % Print the title page as the first slide
\end{frame}

%\begin{frame}
%\frametitle{Overview} % Table of contents slide, comment this block out to remove it
%\tableofcontents % Throughout your presentation, if you choose to use \section{} and \subsection{} commands, these will automatically be printed on this slide as an overview of your presentation
%\end{frame}

%I suggest that you focus on the main points including the following:
%- what is the context of your work and why it is important? (one slide)
%
%- what are the key questions that you are exploring in your paper? (one slide)
%
%- what is the method/approach you used to organize selected studies (i.e. what are the 
%main aspects/metrics/issues you chose to compare and contrast different approaches, 
%in short what criteria you use to organize prior approaches in your report)
%is this your own method or you borrowed it from other places? (one slide)
%
%- What are the main points, findings, lessons learned in an organized (point by point) manner,
%again did you drive these points yourself or they are stated in other studies? (a few slides)

%----------------------------------------------------------------------------------------
%	PRESENTATION SLIDES
%----------------------------------------------------------------------------------------

%------------------------------------------------
%\section{Introduction} % Sections can be created in order to organize your presentation into discrete blocks, all sections and subsections are automatically printed in the table of contents as an overview of the talk
%------------------------------------------------

\section{Background} % A subsection can be created just before a set of slides with a common theme to further break down your presentation into chunks
% 1

\begin{frame}

\begin{itemize}
\item Why atom location?
	\begin{itemize}
	\item Camera in the Steck Lab takes pictures of atoms, but we want to know where its location.
	\end{itemize}
\item Why machine learning? 
	\begin{itemize}
	\item Automatically predict atom location for a new image based on training examples.
	\end{itemize}
\item Why shared memory parallelism?
	\begin{itemize}
	\item Generate finer resolution images faster.
	\item Finer resolution resolution images provide better accuracy. 
	\end{itemize}
\item Why distributed parallelism? 
	\begin{itemize}
	\item Gradient descent is an \emph{embarrassingly parallel} iterative process.
	\item Scalable data $\Rightarrow$ more available parallelism (Gustafson's Law).
	\end{itemize}
\end{itemize}

\end{frame}


\begin{frame}{Input Data}

\begin{figure}
  \centering
  \includegraphics[scale=0.5]{ccd2D.png}
\end{figure}

\end{frame}

\begin{frame}{Input Data}

\begin{figure}
  \centering
  \includegraphics[scale=0.57]{ccd3D.png}
\end{figure}

\end{frame}



\section{Design}

\begin{frame}{Design and Objective}

\begin{figure}[h]
\begin{center}
\includegraphics[scale=0.32]{arch.png}
\caption{High Level Data Flow Architecture Diagram}
\label{fig:small}
\end{center}
\end{figure}

\textbf{Objective:} Create and test a scalable machine learning solution for atom location prediction using CCD images.

\end{frame}


\section{Development}

\begin{frame}{Development and Implementation}

Our development efforts are divided into sections based on our data flow architecture with some overlap of effort in the data preprocessing step.\\

\vspace{.3cm}
\textbf{4 Development Directions:}
\begin{itemize}
\item Simulation
\item Data Preprocessing
\item Machine Learning
\item Visualization
\end{itemize}

\end{frame}

\begin{frame}{Simulation}

\textbf{Raytracer Development}
\begin{itemize}
\item Basic Raytracer -- vector operations, surface definition, refraction
\item Imaging System
\end{itemize}

\begin{figure}
\includegraphics[scale=0.3]{asphere.png}
\caption{Scale model of the atom imaging system.}
\end{figure}

\end{frame}

\begin{frame}{Simulation}

\begin{figure}
\centering
\begin{subfigure}{.5\textwidth}
  \centering
  \includegraphics[scale=.3]{out.png}
  \caption{A norm}
  \label{fig:sub1}
\end{subfigure}%
\begin{subfigure}{.5\textwidth}
  \centering
  \includegraphics[scale=.3]{out2.png}
  \caption{A snell}
  \label{fig:sub2}
\end{subfigure}
\label{fig:test}
\end{figure}


\end{frame}

\begin{frame}{Simulation}
\begin{figure}
\centering
\begin{subfigure}{.5\textwidth}
  \centering
  \includegraphics[scale=.3]{out4.png}
  \caption{A refract}
  \label{fig:sub3}
\end{subfigure}%
\begin{subfigure}{.5\textwidth}
  \centering
  \includegraphics[scale=.3]{out5.png}
  \caption{A lenstest}
  \label{fig:sub4}
\end{subfigure}
\label{fig:test2}
\end{figure}

\end{frame}


\begin{frame}{Data Preprocessing}

\textbf{4 Directions for Data Preprocessing:}
\begin{itemize}
\item Filtering
\item Noise Adding
\item Data Partitioning
\item Feature Scaling
\end{itemize}

\end{frame}

\begin{frame}{Machine Learning}
Built distributed logistic regression classifier capable of multi-class classification and mini-batch processing:

\begin{itemize}
\item Data partitioning and feature scaling handled automatically
\item MPI with Allreduce and custom reduce fuctions
\item Iterative performs gradient updates and reduces updates using summation
\end{itemize}

Unsuccessful attempt to use distributed SVM implementation to compare with our own classifier.

\end{frame}


\begin{frame}{Data Visualization}
Used Mathematica to create contour plots to visual images and plots to demonstrate accuracy of the classifier.

\begin{figure}
\centering
\begin{subfigure}{.5\textwidth}
  \centering
  \includegraphics[scale=.2]{2-5.pdf}
\end{subfigure}%
\begin{subfigure}{.5\textwidth}
  \centering
  \includegraphics[scale=.2]{4-3.pdf}
\end{subfigure}
\label{fig:test2}
\end{figure}

\end{frame}

\begin{frame}{Data Visualization}

\begin{figure}[h]
\begin{center}
\includegraphics[scale=0.26]{2d.pdf}
\end{center}
\end{figure}

\end{frame}



\section{Experiments}

\begin{frame}{Experiments}
Trained and on two data, one small and one large to test scalability.  Prediction Results are 100\% accurate after only 10 iterations.

\begin{figure}[h]
\begin{center}
\includegraphics[scale=0.4]{small_metrics.png}
\caption{Performance gains for both data loading and iteration time on the small training set.}
\label{fig:small}
\end{center}
\end{figure}

\end{frame}

\begin{frame}{Experiments}

\begin{figure}[h]
\begin{center}
\includegraphics[scale=0.5]{big_metrics.png}
\caption{Performance gains for both data loading and iteration time on the big training set.}
\label{fig:small}
\end{center}
\end{figure}

\end{frame}


\section{Conclusion}

\begin{frame}{Conclusion}

\begin{itemize}
\item Implemented parallelized raytracer and generated lots of images
\item Preprocessed images to create training and testing sets
\item Successfully implemented and tested distributed logistic regression classifier using our training and testing dataset
\item Visualized out results to prove classifier accuracy and performance
\end{itemize}

\end{frame}


%------------------------------------------------

\begin{frame}
\Huge{\centerline{The End}}
\end{frame}

%---------------------------------------------------------------------------------------

%\begin{frame}[allowframebreaks]{References}
%\begin{small}
%\bibliographystyle{plainnat}
%\bibliography{references}
%\end{small}
%\end{frame}

\end{document} 